% Options for packages loaded elsewhere
\PassOptionsToPackage{unicode}{hyperref}
\PassOptionsToPackage{hyphens}{url}
%
\documentclass[11pt
]{article}
\usepackage{amsmath,amssymb}
\usepackage{lmodern}
\usepackage{iftex}
\usepackage[letterpaper,top=2cm,bottom=2cm,left=2cm,right=2cm,marginparwidth=1.75cm]{geometry}
\ifPDFTeX
  \usepackage[T1]{fontenc}
  \usepackage[utf8]{inputenc}
  \usepackage{textcomp} % provide euro and other symbols
\else % if luatex or xetex
  \usepackage{unicode-math}
  \defaultfontfeatures{Scale=MatchLowercase}
  \defaultfontfeatures[\rmfamily]{Ligatures=TeX,Scale=1}
\fi
% Use upquote if available, for straight quotes in verbatim environments
\IfFileExists{upquote.sty}{\usepackage{upquote}}{}
\IfFileExists{microtype.sty}{% use microtype if available
  \usepackage[]{microtype}
  \UseMicrotypeSet[protrusion]{basicmath} % disable protrusion for tt fonts
}{}
\makeatletter
\@ifundefined{KOMAClassName}{% if non-KOMA class
  \IfFileExists{parskip.sty}{%
    \usepackage{parskip}
  }{% else
    \setlength{\parindent}{0pt}
    \setlength{\parskip}{6pt plus 2pt minus 1pt}}
}{% if KOMA class
  \KOMAoptions{parskip=half}}
\makeatother
\usepackage{xcolor}
\usepackage{color}
\usepackage{fancyvrb}
\newcommand{\VerbBar}{|}
\newcommand{\VERB}{\Verb[commandchars=\\\{\}]}
\DefineVerbatimEnvironment{Highlighting}{Verbatim}{commandchars=\\\{\}}
% Add ',fontsize=\small' for more characters per line
\newenvironment{Shaded}{}{}
\newcommand{\AlertTok}[1]{\textcolor[rgb]{1.00,0.00,0.00}{\textbf{#1}}}
\newcommand{\AnnotationTok}[1]{\textcolor[rgb]{0.38,0.63,0.69}{\textbf{\textit{#1}}}}
\newcommand{\AttributeTok}[1]{\textcolor[rgb]{0.49,0.56,0.16}{#1}}
\newcommand{\BaseNTok}[1]{\textcolor[rgb]{0.25,0.63,0.44}{#1}}
\newcommand{\BuiltInTok}[1]{\textcolor[rgb]{0.00,0.50,0.00}{#1}}
\newcommand{\CharTok}[1]{\textcolor[rgb]{0.25,0.44,0.63}{#1}}
\newcommand{\CommentTok}[1]{\textcolor[rgb]{0.38,0.63,0.69}{\textit{#1}}}
\newcommand{\CommentVarTok}[1]{\textcolor[rgb]{0.38,0.63,0.69}{\textbf{\textit{#1}}}}
\newcommand{\ConstantTok}[1]{\textcolor[rgb]{0.53,0.00,0.00}{#1}}
\newcommand{\ControlFlowTok}[1]{\textcolor[rgb]{0.00,0.44,0.13}{\textbf{#1}}}
\newcommand{\DataTypeTok}[1]{\textcolor[rgb]{0.56,0.13,0.00}{#1}}
\newcommand{\DecValTok}[1]{\textcolor[rgb]{0.25,0.63,0.44}{#1}}
\newcommand{\DocumentationTok}[1]{\textcolor[rgb]{0.73,0.13,0.13}{\textit{#1}}}
\newcommand{\ErrorTok}[1]{\textcolor[rgb]{1.00,0.00,0.00}{\textbf{#1}}}
\newcommand{\ExtensionTok}[1]{#1}
\newcommand{\FloatTok}[1]{\textcolor[rgb]{0.25,0.63,0.44}{#1}}
\newcommand{\FunctionTok}[1]{\textcolor[rgb]{0.02,0.16,0.49}{#1}}
\newcommand{\ImportTok}[1]{\textcolor[rgb]{0.00,0.50,0.00}{\textbf{#1}}}
\newcommand{\InformationTok}[1]{\textcolor[rgb]{0.38,0.63,0.69}{\textbf{\textit{#1}}}}
\newcommand{\KeywordTok}[1]{\textcolor[rgb]{0.00,0.44,0.13}{\textbf{#1}}}
\newcommand{\NormalTok}[1]{#1}
\newcommand{\OperatorTok}[1]{\textcolor[rgb]{0.40,0.40,0.40}{#1}}
\newcommand{\OtherTok}[1]{\textcolor[rgb]{0.00,0.44,0.13}{#1}}
\newcommand{\PreprocessorTok}[1]{\textcolor[rgb]{0.74,0.48,0.00}{#1}}
\newcommand{\RegionMarkerTok}[1]{#1}
\newcommand{\SpecialCharTok}[1]{\textcolor[rgb]{0.25,0.44,0.63}{#1}}
\newcommand{\SpecialStringTok}[1]{\textcolor[rgb]{0.73,0.40,0.53}{#1}}
\newcommand{\StringTok}[1]{\textcolor[rgb]{0.25,0.44,0.63}{#1}}
\newcommand{\VariableTok}[1]{\textcolor[rgb]{0.10,0.09,0.49}{#1}}
\newcommand{\VerbatimStringTok}[1]{\textcolor[rgb]{0.25,0.44,0.63}{#1}}
\newcommand{\WarningTok}[1]{\textcolor[rgb]{0.38,0.63,0.69}{\textbf{\textit{#1}}}}
\usepackage{longtable,booktabs,array}
\usepackage{calc} % for calculating minipage widths
% Correct order of tables after \paragraph or \subparagraph
\usepackage{etoolbox}
\makeatletter
\patchcmd\longtable{\par}{\if@noskipsec\mbox{}\fi\par}{}{}
\makeatother
% Allow footnotes in longtable head/foot
\IfFileExists{footnotehyper.sty}{\usepackage{footnotehyper}}{\usepackage{footnote}}
\makesavenoteenv{longtable}
\setlength{\emergencystretch}{3em} % prevent overfull lines
\providecommand{\tightlist}{%
  \setlength{\itemsep}{0pt}\setlength{\parskip}{0pt}}
\setcounter{secnumdepth}{-\maxdimen} % remove section numbering
\ifLuaTeX
  \usepackage{selnolig}  % disable illegal ligatures
\fi
\IfFileExists{bookmark.sty}{\usepackage{bookmark}}{\usepackage{hyperref}}
\IfFileExists{xurl.sty}{\usepackage{xurl}}{} % add URL line breaks if available
\urlstyle{same} % disable monospaced font for URLs
\hypersetup{
  hidelinks,
  pdfcreator={LaTeX via pandoc}}

\author{Yu-Hsin CHEN}
\date{2023/3/25}

\begin{document}

\hypertarget{paper-replication}{%
\section{Paper Replication: What motivates effort? Evidence and Experts Forecast}\label{paper-replication}}

\hypertarget{background}{%
\subsection{Background}\label{background}}

This paper are discussing what motivates people to make effort and it is
based on an online experiment, which asked participants to press the
button as much as they can within a limited time. When timeout, the
participants would be given a reward according to how many times they press
the button and also according to the treatment they are allocated. In
the experiment, participants would not know there are several treatments
and it is also not allowed to re-access or refresh the experiment page.

\hypertarget{treatment}{%
\subsection{Treatment}\label{treatment}}

Following treatment is allocated to participants and press "a" and "b" once
count as 1 point.

\begin{itemize}
\item
  Benchmark Treatment

  \begin{enumerate}
  \def\labelenumi{\arabic{enumi}.}
  \item
    Normal Piece Rate: Participants would earn, \textbf{1 cent per 100
    points} or \textbf{10 cents per 100 points}
  \item
    Special Piece Rate: Participants would earn, \textbf{1 cent per 1000
    points} or \textbf{no reward}
  \end{enumerate}
\item
  Social preferences

  \begin{enumerate}
  \def\labelenumi{\arabic{enumi}.}
  \item
    Charity: \textbf{A Charity organization} would earn, \textbf{1 cent
    per 100 points} or \textbf{10 cents per 100 points}
  \item
    Gift exchange: Participants would get \textbf{40 cents} anyway and
    the score would not affect the payment
  \end{enumerate}
\item
  Discounting

  \begin{enumerate}
  \def\labelenumi{\arabic{enumi}.}
  \item
    Participants would earn 1 cent per 100 points after \textbf{two weeks
    from today} or \textbf{four weeks from today}, in contrast,
    participants for other treatments would receive the reward within 24
    hours(if they have a reward).
  \end{enumerate}
\item
  Gain versus losses

  \begin{enumerate}
  \def\labelenumi{\arabic{enumi}.}
  \item
    Participants would be \textbf{paid extra 40 cents} if they
    \textbf{score over 2000} points
  \item
    Participants would be told that they would \textbf{lose extra 40
    cents bonus} if they did not reach the 2000 points threshold
  \item
    Participants would be \textbf{paid extra 80 cents} if they
    \textbf{score over 2000}
  \end{enumerate}
\item
  Risk aversion and probability weighting

  \begin{enumerate}
  \def\labelenumi{\arabic{enumi}.}
  \item
    Participants would have either \textbf{1\% chance} to be paid extra
    1 dollar for every 100 points or \textbf{50\% chance} to be paid
    extra 2 cents for every 100 points
  \end{enumerate}
\item
  Social comparison: Participants would be told that many of others can
  score more than 2000 points and no reward would be paid
\item
  Ranking: Result of participants would be compared with the result from
  other people and no reward would be paid
\item
  Task significance: Participants would be asked to do their very best and
  no reward would be paid
\end{itemize}

\hypertarget{result}{%
\subsection{Result}\label{result}}

\hypertarget{information-obtaining-from-the-original-data}{%
\subsubsection{Information obtaining from the original
data}\label{information-obtained-from-the-original-data}}

Please find all the figures in the following path:
\textbf{/bld/python/figures}

\begin{itemize}
\item
  From the graph \textbf{confidence\_interval}, we can notice that
  \textbf{The No treatment} has the lowest number of button press and \textbf{gain
  80 cents if score over 2000 points} has the highest number of button press
\item
  From the graph \textbf{benchmark\_treatment}, 1.1 is \textbf{1 cent
  per 100 points}, 1.2 is \textbf{10 cents per 100 points}, 1.3 is
  \textbf{No payment}, 2 is \textbf{1 cent per 1000 points}. Here we can
  notice that relatively many participants whose treatment is
  \textbf{no payment} score lower than 1000 compareed with participants
  from other treatments. Additionally, there is a huge difference between
  1.3 and 2, which means that even the participants only receive very
  few rewards, it is still much more motivated than no rewards.
\item
  From the graph \textbf{charity treatment}, 1 cent per 100 points(1.1)
  can be regards as a benchmark. In the chart, notice that participants
  are more willing to making effort when receiving the reward by themselves, compared
  to donating to the charity organization. Besides, the difference between
  3.1, \textbf{donating 1 cent per 100 points}, and 3.2,
  \textbf{donating 10 cents per 100 points}, are relatively small.
  Therefore, we can say that participants are more care about if they
  donate the reward, instead of how much they donate.
\item
  From the graph \textbf{probability weight treatment}, the benchmark
  treatment of \textbf{1 cent per 100 points(1.1)} seems a more attractive
  treatment, compared to the low probability but high reward option, which
  is \textbf{1\% chance to be paid extra 1 dollar per 100 points (6.1)}
  and \textbf{50\% chance to be paid extra 2 cents per 100 points(6.2)}.
\item
  From the graph \textbf{delay treatment}, we can observe that the
  \textbf{benchmark treatment 1.1} is just slightly more motivated them
  receiving the same amount but \textbf{two weeks delay rewards(4.1)} and \textbf{four weeks
  delay rewards(4.2)} and we can also hardly tell the difference between
  4.1 and 4.2.
\item
  From the graph \textbf{gain loss treatment}, we can observe that all
  the gain loss treatments, \textbf{being paid extra 40 cents if they
  score over 2000 points}, 5.1, \textbf{losing extra 40 cents bonus if
  they score below 2000 points}, 5.2, and \textbf{being paid extra 80
  cents if they score over 2000 points}, 5.3, are more motivated than
  benchmark 1.1. Moreover, 5.2 and 5.3 has very similar motivation
  affect. Besides, there are very few participants finished their task
  below 2000 points, and after exceeding the threshold,2000, number of
  participants increase immediately.
\item
  In the graph of \textbf{social treatment}, including social
  \textbf{comparison(7)}, \textbf{ranking(8)}, and \textbf{task significance(9)}, all of these
  treatments are clearly less motivated than the benchmark (1.1).
\end{itemize}

\hypertarget{information-obtained-from-table}{%
\subsubsection{Information obtained from
table}\label{information-obtaining-from-table}}

\textbf{All the table result} please find in the following path:
\textbf{\textbackslash bld\textbackslash python\textbackslash data}\\
\textbf{All the graph result} please find in the following path:
\textbf{\textbackslash bld\textbackslash python\textbackslash figures}

The assumption of exponential cost function and power cost function
please check the original paper \emph{What motivates effort? Evidence
and Experts Forecast} This article would not go deep into the technical part.
The \textbf{table\_1} in the above table path, shows the following
result and it is the estimation result of these three parameters in the
\textbf{power cost function}.

\begin{longtable}[]{@{}lllll@{}}
\toprule()
Parameters & curve fit & least squared & minimized\_nd & authors \\
\midrule()
\endhead
Curvature of cost function & 21.1939 & 21.787 & 21.266 & 20.546 \\
Level K of cost effort & 5.92E-72 & 2.17E-69 & 3.45E-72 & 5.12E-70 \\
Intrinsic motivation S & 1.38E-06 & 9.56E-08 & 1.33E-06 & 3.17E-06 \\
Min Objective function & 670.61 & 1112.339 & 670.61 & 672.387 \\
\bottomrule()
\end{longtable}

Here, all the results are similar to the original code, but
in order to make the scale comparable to the result from the paper, I
change the scale to which same as the one in the paper.

The \textbf{table\_1\_r2} in the above table path shows the following
result. The \textbf{table\_1\_r2} result is for the \textbf{exponential
cost function}

\begin{longtable}[]{@{}lllll@{}}
\toprule()
Parameters & curve fit & least squared & minimized\_nd & authors \\
\midrule()
\endhead
Curvature of cost function & 0.0156 & 0.011 & 0.016 & 0.016 \\
Level K of cost effort & 1.71E-16 & 8.22E-12 & 2.71E-16 & 1.71E-16 \\
Intrinsic motivation S & 3.72E-06 & 1.16E-04 & 3.72E-06 & 3.72E-06 \\
\bottomrule()
\end{longtable}

The \textbf{table\_1\_r2} is not in the original python code, but in
order to keep the consistency of the content, it has been added. The
\textbf{table\_1} and \textbf{table\_1\_r2} apply curve fit, least
squared, and minimized\_nd, three kinds of estimation method.

\begin{Shaded}
\begin{Highlighting}[]
\ImportTok{import}\NormalTok{ scipy.optimize }\ImportTok{as}\NormalTok{ opt
}
\CommentTok{\#following are the method I used
}
\NormalTok{opt.curve\_fit
}
\NormalTok{opt.least\_squares
}
\NormalTok{opt.minimize}
\end{Highlighting}
\end{Shaded}

These two tables are the replication of \textbf{table5 }of the NLS part in
PANEL A in the paper. Because the objective function results for the exponential
cost function is relatively large, so I did not include it in
\textbf{table\_1\_r2}. It is probably the result of the Exp cost function
per se or because there is something wrong in the estimation process,
and this table is not in the original code, the result cannot be
compared. However, the parameters estimation result is close to the
result from the authors anyway.\\
Additionally, I also convert these tables to graphs to make it
easier for comparison. Please check, \textbf{method\_comparison} in the
figure file.

Next, the \textbf{table\_2} in the data file shows the following result.

\begin{longtable}[]{@{}lllllll@{}}
\toprule()
parameters & power\_est & power\_aut & se\_p & exp\_est & se\_e &
exp\_aut \\
\midrule()
\endhead
Curvature of cost function & 21.1939 & 20.546 & 7.39896 & 0.0156 &
0.00415 & 0.0156 \\
Level k of cost of effort & 5.95E-72 & 5.12E-70 & 3.33E-70 & 1.71E-16 &
1.49E-15 & 1.71E-16 \\
Intrinsic motivation s & 1.38E-06 & 3.17E-06 & 4.93E-06 & 3.72E-06 &
9.16E-06 & 3.72E-06 \\
Social preference & 0.013 & 0.006 & 0.03 & 0.004 & 0.011 & 0.07 \\
Warm glow coefficient & 0.265 & 0.182 & 0.288 & 0.143 & 0.143 & 0.035 \\
Gift exchange coefficient & 3.25E-05 & 2.04E-05 & 8.14E-05 & 2.35E-05 &
4.82E-05 & 3.00E-05 \\
present biased coefficient & 1.616 & 1.36 & 2.053 & 1.237 & 1.302 &
0.79 \\
Weekly discount factor & 0.751 & 0.75 & 0.292 & 0.754 & 0.239 & 0.86 \\
\bottomrule()
\end{longtable}

This table is the replication of \textbf{Table 5} NLS part in PANEL B
in the paper. \textbf{power\_est} is the parameters estimation result for
the power cost function, which applies the curve fit method. \textbf{power\_aut}
are parameters of the power cost function from the authors. \textbf{se\_p}
is their standard error. Furthermore, \textbf{exp\_est} are parameters
estimation results for exponential cost function. \textbf{exp\_aut} is
also the result of authors for the Exp cost function, and \textbf{se\_e} is
still the standard error. Besides, I also convert the result to 
graphs for easier comparison. Please check
\textbf{params\_comparison\_exp} and \textbf{params\_comparison\_power}
in the figures file.

Next, the \textbf{table\_3} in the data file shows the following result

{\setlength\tabcolsep{2.5pt}\small
\begin{longtable}{l*{10}{r}}
%\begin{longtable}[]{@{}llllllllll@{}}
\toprule()
parameters & P\_est1 & P\_se1 & P\_aut1 & P\_est2 & P\_se2 & P\_aut2 &
P\_est3 & P\_aut3 & P\_se3 \\
\midrule()
\endhead
Curvature of cost function & 20.9475 & 5.7752 & 20.59 & 18.9623 & 5.2711
& 18.87 & 19.6386 & 19.64 & 17.3201 \\
Level k of cost of effort & 3.89E-71 & 1.70E-69 & 3.77E-70 & 1.96E-64 &
7.68E-63 & 3.92E-64 & 1.01E-66 & 1.02E-66 & 1.36E-64 \\
Intrinsic motivation s & 1.57E-06 & 4.23E-06 & 2.66E-06 & 5.96E-06 &
1.47E-05 & 6.22E-06 & 3.75E-06 & 3.75E-06 & 4.17E-05 \\
Probability Weighting & 0.1928 & 0.173 & 0.19 & 0.3732 & 0.3042 & 0.38 &
0.2952 & 0.3 & 1.5657 \\
Curvature of utility over piece rate & 1 & 0 & 1 & 0.88 & 0 & 0.88 &
0.9235 & 0.92 & 0.9289 \\
\bottomrule()
\end{longtable}


and

\begin{longtable}[]{@{}llllllllll@{}}
\toprule()
parameters & E\_est4 & E\_se4 & E\_aut1 & E\_est5 & E\_se5 & E\_aut2 &
E\_est6 & E\_aut3 & E\_se6 \\
\midrule()
\endhead
Curvature of cost function & 0.0134 & 0.0026 & 0.01 & 0.0119 & 0.0023 &
0.01 & 0.0072 & 0.01 & 0.0029 \\
Level k of cost of effort & 2.42E-14 & 1.29E-13 & 2.42E-14 & 7.50E-13 &
3.56E-12 & 7.50E-13 & 5.46E-08 & 5.46E-08 & 3.70E-07 \\
Intrinsic motivation s & 1.64E-05 & 2.40E-05 & 1.65E-05 & 5.55E-05 &
7.20E-05 & 5.55E-05 & 3.14E-03 & 3.14E-03 & 7.50E-03 \\
Probability Weighting & 0.239 & 0.1427 & 0.24 & 0.466 & 0.2472 & 0.466 &
4.2961 & 4.3 & 5.4624 \\
Curvature of utility over piece rate & 1 & 0 & 1 & 0.88 & 0 & 0.88 &
0.4679 & 0.47 & 0.2368 \\
\bottomrule()
\end{longtable}

The \textbf{table\_3} is the replication of \textbf{table 6} in the
paper. The first table is for the power cost function, and the second one is
for the exponential cost function. The difference between P\_est1, P\_est2
and P\_est3 is their curvature of utility over piece rate. For P\_est1
and P\_est2, it is pre-defined, which is 1 and 0.88 respectively. For
P\_est3, the curvature of utility over piece rate is regarded as a
parameter and curve fit method is applied to find its optimal value,
here is 0.9235.\\
The case for the Exp cost function is the same. The Curvature of utility
over piece rate has been pre-defined as 1 and 0.88 for E\_est4 and
E\_est5. E\_est6 has the optimal Curvature of utility over piece rate,
0.4679. P\_se and E\_se are their standard error. E\_aut and P\_aut are
the results from authors. The tables have been covert to graphs for easier
comparison. Please check \textbf{another\_params\_comparison\_exp} and
\textbf{another\_params\_comparison\_power} in
\textbf{bld/python/figures}

Additionally, the following table, \textbf{table\_gmm} also in the
\textbackslash data file, and it shows following result. Notice that the
result can be slightly different each times because of randomness.

\begin{longtable}[]{@{}lllllll@{}}
\toprule()
parameters & P\_esti & P\_authors & P\_se & E\_esti & E\_authors &
E\_se \\
\midrule()
\endhead
Level k of cost of effort & 2.54E-112 & 2.56E-112 & 1.53E-60 & 1.27E-16
& 1.27E-16 & 2.75E-11 \\
Curvature of cost function & 33.138 & 33.137 & 12.214 & 0.016 & 0.016 &
0.006 \\
Intrinsic motivation s & 7.12E-07 & 7.12E-07 & 1.20E-05 & 3.32E-06 &
3.32E-06 & 2.63E-05 \\
Social Preference & 0.003 & 0.003 & 0.014 & 0.003 & 0.003 & 0.014 \\
Warm glow coefficient & 0.125 & 0.125 & 0.147 & 0.143 & 0.143 & 0.15 \\
Gift exchange coefficient & 3.26E-06 & 3.26E-06 & 2.45E-05 & 8.58E-06 &
8.59E-06 & 3.70E-05 \\
present biased coefficient & 1.17 & 1.17 & 3.05 & 1.15 & 1.15 & 12.64 \\
Weekly discount factor & 0.75 & 0.75 & 0.34 & 0.76 & 0.76 & 0.32 \\
\bottomrule()
\end{longtable}

This table is the replication of \textbf{TABLE 5} panel A and panel B
for the \textbf{minimum distance} method in the paper. As we can see, the
estimation result is identical to the result of authors for power cost
function, and for the Exp cost function, the result just slightly different.
For easier comparison reason, please check the graph
\textbf{mini\_dist\_params\_comparison\_exp} and
\textbf{mini\_dist\_params\_comparison\_power} in the figures file.

\end{document}

% \documentclass[11pt, a4paper, leqno]{article}
% \usepackage{a4wide}
% \usepackage[T1]{fontenc}
% \usepackage[utf8]{inputenc}
% \usepackage{float, afterpage, rotating, graphicx}
% \usepackage{epstopdf}
% \usepackage{longtable, booktabs, tabularx}
% \usepackage{fancyvrb, moreverb, relsize}
% \usepackage{eurosym, calc}
% % \usepackage{chngcntr}
% \usepackage{amsmath, amssymb, amsfonts, amsthm, bm}
% \usepackage{caption}
% \usepackage{mdwlist}
% \usepackage{xfrac}
% \usepackage{setspace}
% \usepackage[dvipsnames]{xcolor}
% \usepackage{subcaption}
% \usepackage{minibox}
% % \usepackage{pdf14} % Enable for Manuscriptcentral -- can't handle pdf 1.5
% % \usepackage{endfloat} % Enable to move tables / figures to the end. Useful for some
% % submissions.

% {\setlength\tabcolsep{1.5pt}}

% parameters & E\_est4 & E\_se4 & E\_aut1 & E\_est5 & E\_se5 & E\_aut2 &
% E\_est6 & E\_aut3 & E\_se6 \\








% \usepackage[
%     natbib=true,
%     bibencoding=inputenc,
%     bibstyle=authoryear-ibid,
%     citestyle=authoryear-comp,
%     maxcitenames=3,
%     maxbibnames=10,
%     useprefix=false,
%     sortcites=true,
%     backend=biber
% ]{biblatex}
% \AtBeginDocument{\toggletrue{blx@useprefix}}
% \AtBeginBibliography{\togglefalse{blx@useprefix}}
% \setlength{\bibitemsep}{1.5ex}
% \addbibresource{../../paper/refs.bib}

% \usepackage[unicode=true]{hyperref}
% \hypersetup{
%     colorlinks=true,
%     linkcolor=black,
%     anchorcolor=black,
%     citecolor=NavyBlue,
%     filecolor=black,
%     menucolor=black,
%     runcolor=black,
%     urlcolor=NavyBlue
% }


% \widowpenalty=10000
% \clubpenalty=10000

% \setlength{\parskip}{1ex}
% \setlength{\parindent}{0ex}
% \setstretch{1.5}


% \begin{document}

% \title{epp final\thanks{Yu Hsin Chen, Bonn University. Email: \href{mailto:s6ynche2@uni-bonn.de}{\nolinkurl{s6ynche2 [at] uni-bonn [dot] de}}.}}

% \author{Yu Hsin Chen}

% \date{
%     {\bf Preliminary -- please do not quote}
%     \\[1ex]
%     \today
% }

% \maketitle


% \begin{abstract}
%     Some abstract here.
% \end{abstract}

% \clearpage


% \section{Introduction} % (fold)
% \label{sec:introduction}

% If you are using this template, please cite this item from the references:
% \citet{GaudeckerEconProjectTemplates}.

% I want to try and see
% The data set for the example project is taken from
% \url{https://www.stem.org.uk/resources/elibrary/resource/28452/large-datasets-stats4schools}.
% It contains data on smoking habits in the UK, with 1691 observations and 12 variables.
% We consider only 4 of the 12 features for the prediction of the variable
% \texttt{smoking}: \texttt{marital\_status}, \texttt{highest\_qualification},
% \texttt{gender} and \texttt{age}. We model the dependence using a Logistic model. All
% numerical features are included linearly, while categorical features are expanded into
% dummy variables. Figures below illustrate the model predictions over the lifetime. You
% will find one figure and one estimation summary table for each installed programming
% language.



% \begin{figure}[H]

%     \centering
%     \includegraphics[width=0.85\textwidth]{../python/figures/smoking_by_marital_status}

%     \caption{\emph{Python:} Model predictions of the smoking probability over the
%         lifetime. Each colored line represents a case where marital status is fixed to one
%         of the values present in the data set.}
%     \label{fig:python-predictions}

% \end{figure}


% \begin{table}[!h]
%     \input{../python/tables/estimation_results.tex}
%     \caption{\label{tab:python-summary}\emph{Python:} Estimation results of the
%         linear Logistic regression.}
% \end{table}




% % section introduction (end)



% \setstretch{1}
% \printbibliography
% \setstretch{1.5}


% % \appendix

% % The chngctr package is needed for the following lines.
% % \counterwithin{table}{section}
% % \counterwithin{figure}{section}
% \end{document}
